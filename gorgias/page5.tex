
\subsection*{Inutilité de la rhétorique}

Le sujet de la discussion est désormais la question de savoir \emph{qui est heureux et qui ne l'est pas}. C'est selon Socrate la question sur laquelle il est le plus beau de savoir la vérité et le plus honteux de l'ignorer.

%Aux yeux de Polos, l'opinion de Socrate est ridicule et facile à réfuter. Socrate prétend au contraire que cette opinion est irréfutable, car c'est, dit-il, la vérité même. Il y a une opposition diamétrale entre Polos et Socrate, et cette opposition porte à la fois sur la question de savoir qui est heureux ou malheureux, et sur la question de savoir ce qu'est une réfutation.

\begin{comment}
POLOS : Voilà, Socrate, une étrange théorie.\\
SOCRATE : Je vais essayer pourtant, mon ami, de te la faire partager avec moi ; car je te considère comme mon ami. Pour le moment, la différence qui nous sépare est celle-ci~ : j'ai dit au cours de notre entretien que commettre l'injustice était pire que la subir.\\
POLOS : Oui.\\
SOCRATE : Et toi, que la subir était pire.\\
POLOS : Oui.\\
SOCRATE : J'ai dit aussi que les coupables étaient malheureux, et tu as réfuté mon affirmation.\\
POLOS : Assurément, par Zeus.\\
SOCRATE : C'est du moins ton opinion.\\
POLOS : Et une opinion qui n'est point fausse !\\
SOCRATE : Peut-être. Toi, au contraire, tu juges heureux les coupables qui échappent au châtiment.\\
POLOS : Sans aucun doute.\\
SOCRATE : Moi, je prétends que ce sont les plus malheureux, et que ceux qui expient le sont moins. Veux-tu réfuter aussi cette partie de ma thèse ?\\
POLOS : Seconde réfutation encore plus difficile, en vérité, que la première, Socrate !\\
SOCRATE : Ne dis pas difficile, Polos, mais impossible ; car la vérité est irréfutable.\\
POLOS : Que dis-tu là ? Voici un homme qui est arrêté au moment où il essaie criminellement de renverser un tyran ; aussitôt pris, on le torture, on lui coupe des membres, on lui brûle les yeux, et après qu'il a été soumis lui-même à mille souffrances atroces,
après qu'il a vu ses enfants et sa femme livrés aux mêmes supplices, on finit par le mettre en croix ou l'enduire de poix et le brûler vif : et cet homme, il serait plus heureux de la sorte que s'il avait pu s'échapper, devenir tyran, gouverner la cité toute sa vie en se livrant à tous ses caprices, objet d'envie et d'admiration pour les citoyens et pour les étrangers ?
Voilà la thèse que tu dis irréfutable ?\\
SOCRATE : Tu me présentes un épouvantail, brave Polos, non une réfutation, pas plus que tout à l'heure avec tes témoins. Quoiqu'il en soit, veuille me rappeler un détail ; tu as bien dit~: « au moment où il essaie criminellement de renverser un tyran ? »\\
POLOS : Oui.\\
SOCRATE : Dans ce cas, il ne saurait y avoir aucune supériorité de bonheur ni pour celui qui s'empare de la tyrannie injustement ni pour celui qui est livré au châtiment ; car, de deux malheureux, ni l'un ni l'autre n'est « le plus heureux ». Ce qui est vrai,
c'est que le plus malheureux des deux est celui qui a pu échapper à la justice et devenir tyran. Quoi, Polos ? Tu ricanes ?
Est-ce là encore une nouvelle forme de réfutation, que de se moquer de ce qu'on dit, sans donner de raisons ?
\end{comment}

%Cependant, Socrate et Polos finissent par s'accorder sur un point~: c'est que commettre l'injustice est \emph{plus laid} que de la subir. À partir de ce point concédé par Polos, Socrate montre que commettre l'injustice est aussi plus mauvais. Car ce qui est beau est nécessairement ou agréable ou utile, et inversement ce qui est laid est ou pénible ou nuisible. Or il est moins pénible de commettre l'injustice que de la subir. Il faut donc que ce soit plus mauvais. De même, le juste châtiment que reçoit l'homme qui paie sa faute est beau, quoique pénible~: il est donc bon.

Le criminel qui n'est pas puni pour ses crimes est le plus malheureux des hommes : il est comme un malade qui ne serait pas soigné. La justice est l'art qui nous délivre du plus grand des maux : elle est pour l'âme ce que la médecine est pour le corps. Par conséquent, la rhétorique, qui permet au coupable d'échapper au châtiment, n'a aucune utilité.

À la rigueur, le seul bon usage qu'on pourrait faire de la rhétorique serait de s'accuser soi-même lorsqu'on a commis une faute.
