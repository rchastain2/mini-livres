\subsection*{Bien employer les jours que nous avons à vivre}

Face à l'opiniâtreté de Calliclès, Socrate en est réduit à continuer la discussion tout seul.

L'homme tempérant et sage, dit-il, se conduit envers les dieux et envers les hommes de la manière qui convient. Agir comme il convient à l'égard des hommes, c'est être juste. Agir comme il convient à l'égard des dieux, c'est être pieux. L'homme sage ne se laisse détourner de ses devoirs, ni par le plaisir, ni par la peine : il est donc aussi courageux.

Quant au reproche que Calliclès fait à Socrate, sur son incapacité à se défendre contre l'injustice qu'il pourrait lui arriver de subir, que faut-il en penser ? Si le plus grand mal qui puisse nous arriver était de subir l'injustice, il faudrait avant tout chercher à se rendre fort : puisque c'est la force qui nous met à l'abri de l'injustice que nous pourrions subir.% Le mieux sera de prendre le pouvoir dans la cité, si possible même un pouvoir tyrannique ; ou au moins d'être un ami du gouvernement existant. Pour être l'ami du tyran, il faudra lui ressembler, aimer et blâmer les mêmes choses que lui.

%D'ailleurs, la rhétorique n'est pas le seul art qui nous sauve du péril. La natation, la navigation, la médecine, le font aussi bien. Pourquoi faudrait-il avoir plus de considération pour l'orateur que pour le maître-nageur, le pilote ou le médecin ?
%
%En réalité, la vie, sa durée plus ou moins longue, ne méritent pas de préoccuper un homme vraiment homme~; au lieu de s'attacher à elle avec amour, il faut s'en remettre à la divinité du soin de régler ces choses. Il faut plutôt se préoccuper de savoir comment nous emploierons les jours que nous avons à vivre. Voulons-nous être bien vu et acquérir du crédit dans la cité~? Il faut nous adapter à la constitution politique du pays dans lequel nous vivions. Si c'est une démocratie, il faudra penser comme le peuple, pour se faire aimer de lui et devenir grand dans la cité.
%
%Cependant la véritable tâche de l'homme d'État n'est pas de plaire aux citoyens, mais de faire leur bien, de les rendre meilleurs, c'est-à-dire bons, justes, tempérants, raisonnables.
%
%Quant à ce que Calliclès ne cesse de répéter, que Socrate pourrait être injustement accusé et condamné à mort, faute de connaître la rhétorique, Socrate admet qu'une telle chose pourrait bien arriver, et qu'elle n'aurait même rien d'étonnant. En effet, Socrate est l'un des rares Athéniens, pour ne pas dire le seul, qui cultive le véritable art politique. Il ne cherche jamais à plaire par son langage~; il a toujours en vue le bien, et non pas l'agréable. S'il était accusé, il serait comme un médecin traduit devant un tribunal d'enfants par un cuisinier.
