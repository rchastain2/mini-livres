
\subsection*{La nature de la rhétorique selon Gorgias}

%Gorgias est de passage à Athènes, pour y donner une conférence. Il invite l'auditoire à lui poser n'importe quelle question, à laquelle il promet de répondre par un discours improvisé. 

Socrate interroge Gorgias sur \emph{ce qu'il est} : il voudrait connaître le nom et la nature de l'art que Gorgias enseigne.

Cet art, c'est la rhétorique. C'est l'art de persuader des citoyens réunis en assemblée, au sujet ce qui est juste et ce qui ne l'est pas. L'orateur, reconnaît honnêtement Gorgias, n'apporte aucun savoir à ses auditeurs ; il ne produit dans leur esprit qu'une simple croyance, vraie ou fausse. Cependant, pour faire bon usage de son art, il doit ou posséder lui-même la science des choses dont il parle, ou la tenir d'un autre.

Il est arrivé à Gorgias d'accompagner un médecin chez un patient qui ne voulait pas se laisser amputer ou cautériser. Tandis que le médecin était impuissant à persuader son patient de se soumettre au traitement, Gorgias y parvenait, par la seule connaissance de l'art oratoire.

%Gorgias est l'orateur honnête, qui met son art au service du bien et de la vérité.
