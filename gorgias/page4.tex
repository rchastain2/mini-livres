\subsection*{Polos ou l'amour du pouvoir}

% Polos, un jeune élève de Gorgias, a peine à se contenir lorsqu'il entend Socrate développer sa théorie concernant la rhétorique.

Les orateurs, objecte Polos, sont tout-puissants dans la cité : ils peuvent y faire condamner à mort, à la prison ou à l'exil qui ils veulent.

Sans doute les orateurs font tout ce qui leur plaît, mais ils ne font pas pour autant ce qu'ils veulent. Ce que nous voulons, dans toutes nos actions, ce n'est pas l'action elle-même, mais c'est sa fin ; et la fin de toutes nos actions est le bien.

% Par conséquent, celui qui fait \emph{ce qui lui plaît} ne fait pas nécessairement \emph{ce qu'il veut}. L'action qu'il estime avantageuse pour lui peut tourner à son préjudice ; auquel cas il aurait mieux valu pour lui ne pas pouvoir accomplir cette action. Il est donc clair que cette puissance dont parle Polos, cette liberté de faire ce qui nous plaît, n'est pas un bien en soi : c'est une chose tantôt bonne, tantôt mauvaise.

N'est-il pas enviable, insiste Polos, l'homme qui agit à sa guise dans la cité, qui fait tuer, dépouiller ou jeter en prison qui il lui plaît, justement ou injustement ?

Cet homme n'est enviable ni dans un cas ni dans l'autre. Il est même à prendre en pitié, s'il fait tout cela de manière injuste.% Car le plus grand des maux, c'est de commettre l'injustice.

\begin{comment}
POLOS : Comme si toi-même, Socrate, tu n'aimerais pas mieux avoir la liberté de faire dans l'État ce qui te plairait que d'en être empêché, et comme si, en voyant un homme tuer, dépouiller, mettre aux fers qui il lui plairait, tu ne lui portais pas envie !\\
SOCRATE : Entends-tu qu'il agirait justement ou injustement ?\\
POLOS : De quelque manière qu'il agisse, ne serait-il pas enviable dans un cas comme dans l'autre ?\\
SOCRATE : Ne parle pas ainsi, Polos.\\
POLOS : Pourquoi donc ?\\
SOCRATE : Parce qu'il ne faut pas envier les gens qui ne sont pas enviables, non plus que les malheureux, mais les prendre en pitié.
\end{comment}

L'homme le plus à plaindre, n'est pas celui qui subit l'injustice, mais celui qui la commet ; et il est plus malheureux encore s'il ne paie point ses fautes et échappe au châtiment qu'il mérite.% Pourtant, ce malheureux parmi les malheureux, c'est celui qu'une part de nous-mêmes envie peut-être en secret ; cette part de nous-mêmes dont Calliclès sera le porte-parole plus loin dans la discussion, lorsque Polos aura à son tour renoncé à tenir tête à Socrate.

% Pour Polos, l'homme qui est le plus à plaindre, c'est celui qui est victime de l'injustice, celui qu'on fait tuer injustement par exemple. Certes, dit Socrate, c'est un malheur, mais un malheur moins grand 1° que de tuer injustement et 2° que d'être tué justement.

\begin{comment}
POLOS : C'est sans doute celui qui meurt injustement qui est digne de pitié et malheureux ?\\
SOCRATE : Moins que celui qui le tue, Polos, et moins que celui qui meurt justement.
\end{comment}
