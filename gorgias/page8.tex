\subsection*{La mort, le jugement}

Mais la mort n'a rien d'effrayant, pour celui qui n'a aucune faute à se reprocher, en paroles ou en actes, ni envers les dieux, ni envers les hommes. Le simple fait de mourir n'a en soi rien d'effrayant~: ce qu'on redoute en fait, c'est d'être coupable au moment de mourir. Socrate propose de raconter une histoire qui le prouve. Calliclès prendra peut-être cette histoire pour un conte : Socrate, lui, la tient pour une histoire vraie.

Aux hommes qui ont mené une existence juste et sainte, il est permis d'aller, après leur mort, dans les îles des Bienheureux, où ils séjournent, à l'abri de tout mal, dans une félicité parfaite. Ceux qui au contraire ont vécu dans l'injustice et dans l'impiété sont envoyés dans un lieu d'expiation et de peine qu'on appelle le Tartare.% Autrefois les hommes étaient jugés de leur vivant, le jour de leur mort, par des juges eux-mêmes vivants. Cependant les jugements étaient mal rendus. Aussi Zeus décida-t-il que les hommes seraient jugés après leur mort. Il destina trois de ses fils, Minos, Rhadamanthe et Éaque, à devenir juges après leur mort. Les trois fils de Zeus rendent désormais leur jugement dans une prairie, à un carrefour d'où partent les deux routes qui mènent, l'une aux îles des Bienheureux, l'autre au Tartare. Rhadamanthe juge les hommes de l'Asie, Éaque ceux de l'Europe. Quant à Minos, il a pour fonction de prononcer en dernier ressort, lorsque ses deux frères sont embarrassés.

%Voilà l'histoire que Socrate a entendu raconter, et qu'il tient pour vraie. La mort~n'est que la séparation de l'âme et du corps. L'âme aussi bien que le corps porte encore, après la mort, les marques que la vie lui a imprimées. Quand le juge examine une âme et qu'il y trouve la marque des iniquités que l'homme a commises de son vivant, il la condamne à être châtiée. Il y a deux sortes de châtiment, l'un qui guérit et l'autre qui sert d'exemple. Ceux qui subissent un châtiment éternel sont le plus souvent des rois ou des hommes puissants. Le reproche que faisait Calliclès à Socrate se retourne contre Calliclès : il pourrait bien être incapable de de défendre, au moment de comparaître devant Éaque.

%~ \subsection*{Conclusion}

%~ Au moment où Platon écrit ce livre, Socrate est mort dans les circonstances que tout le monde connaît : ses ennemis l'ont fait condamner, sur la base de fausses accusations, notamment celle d'avoir professé l'athéisme. La mort de Socrate est donc aussi le sujet du livre. André Dacier, un grand traducteur de Platon, raconte dans une préface que quand Socrate fut condamné, comme on le menait en prison, Apollodore se mit à crier : \emph{Socrate, ce qui m'afflige le plus, c'est de vous voir mourir innocent.} Socrate lui passant doucement la main sur la tête, lui dit en riant : \emph{Mon ami, aimerais-tu mieux me voir mourir coupable ?} C'est déjà la discussion entre Socrate et Polos.

%~ Revenons pour finir sur la similitude notée par Joseph de Maistre aussi bien que par Nietzsche, entre la morale socratique et la morale chrétienne. Plus d'une fois en lisant le Gorgias on croirait lire l'Évangile. Le discours de Socrate, affirmant que celui qui subit l'injustice est moins malheureux que celui qui la commet, fait penser au Sermon sur la montagne, avec ce renversement que note Calliclès.

%~ L'admirable passage sur le pilote, qui ne croit pas mériter un grand salaire, pour avoir fait traverser la mer à ses passagers, parce qu'il ne sait pas si pour certains la mort n'eût pas mieux valu que la vie, ce passage donc ressemble à l'\emph{Évangile selon saint Matthieu} : \emph{Si quelqu'un scandalise un de ces petits qui croient en moi, il vaudrait mieux pour lui qu'on lui pendît au cou une de ces meules qu'un âne tourne, et qu'on le jetât au fond de la mer.} Ou encore, lorsque Socrate, à la toute fin du livre, parle à Calliclès de cette gifle qu'il craint tant et qu'il devrait souffrir, le cas échéant, sans se troubler, on pense à cet autre passage de saint Matthieu : \emph{Et moi je vous dis, de ne point résister au mal que l'on veut vous faire: mais si quelqu'un vous a frappé sur la joue droite, présentez-lui encore l'autre.}
