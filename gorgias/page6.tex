\subsection*{Calliclès, ou l'hédonisme radical}

%Cette étrange théorie fait sortir Calliclès hors des gonds. Socrate plaisante-t-il, ou parle-t-il sérieusement ? Si Socrate est sérieux, et si ce qu'il dit est vrai, observe Calliclès, alors nous faisons tout le contraire de ce qu'il faudrait.

%À en croire Calliclès, il est arrivé à Polos la même chose qu'à Gorgias : il n'a osé dire ce qu'il pensait. Lorsque Polos a concédé que commettre l'injustice était plus laid que la subir, il a parlé selon la loi, non pas selon la nature.

%Socrate se félicite d'avoir trouvé en la personne de Calliclès l'interlocuteur parfait, réunissant trois qualités : le savoir, la bienveillance et la franchise. Ainsi, lorsque Calliclès et Socrate seront d'accord sur un point, cet accord sera une preuve suffisante de vérité.

La véritable justice, dit Calliclès, est que les meilleurs et les plus puissants commandent aux autres et prennent la plus grosse part. Quant à ceux que Socrate appelle les sages, ceux qui se dominent et commandent à leurs passions, ce sont pour Calliclès les imbéciles. Pour être heureux, il faut au contraire, selon Calliclès, entretenir en soi-même les plus fortes passions et leur prodiguer tout ce qu'elles désirent.% Le commun des mortels, qui vante la tempérance et la justice, n'est pas sincère : il ne tient ce discours que pour cacher son impuissance à imiter ceux qu'il envie en secret.

%Si le bonheur, dit encore Calliclès, consistait à n'avoir besoin de rien, il faudrait appeler heureux les pierres et les morts. À quoi Socrate répond que l'homme heureux, tel que Calliclès le conçoit, ressemble à quelqu'un qui passerait sa vie à essayer de remplir d'eau un tonneau percé, en portant l'eau dans une passoire.

%Si le bonheur est la même chose que le plaisir, alors avoir la gale et éprouver des démangeaisons continuelles est une vie heureuse, pourvu qu'on ait le pouvoir et la liberté de se gratter à sa guise. Calliclès reproche alors à Socrate la vulgarité de ses propos. Cependant n'est-ce pas la définition que Calliclès a donnée du bonheur qui a conduit la conversation à ces extrémités ?

En réalité, il est évident que le bonheur et le plaisir sont deux choses différentes. En effet, le bonheur et le malheur sont deux états opposés, comme la santé et la maladie. Le plaisir et la souffrance, au contraire, vont toujours ensemble.% Il est agréable de manger quand on a faim, agréable de boire quand on a soif, c'est-à-dire quand on souffre : car tout besoin, tout désir en soi est pénible. C'est donc que le plaisir n'est pas le bonheur.

%Si le plaisir et le bien sont deux choses différentes, alors l'art et la flatterie sont aussi deux choses différentes, comme Socrate l'avait expliqué à Polos. On peut faire plaisir à quelqu'un sans se soucier de son véritable intérêt. C'est ce que fait souvent le cuisinier ; c'est aussi ce que font souvent le musicien et le poète : ils ne cherchent qu'à plaire au public, sans s'inquiéter de ce qui est bon pour lui. Or la poésie est une espèce de rhétorique : le poète fait au théâtre métier d'orateur.
