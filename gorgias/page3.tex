\subsection*{La nature de la rhétorique selon Socrate}

Il y a donc un bon et un mauvais usage de la rhétorique. C'est la preuve, dit Socrate, que la rhétorique n'est pas un art, car par définition un art a pour objet le bien. La rhétorique n'est qu'un savoir-faire, une habileté à faire plaisir, acquise par expérience. Ce savoir-faire se fait passer pour un art. La rhétorique, s'il faut que Socrate dise toute sa pensée en un mot, est \emph{le fantôme d'une partie de la politique}.

Pour le corps comme pour l'âme, il existe un art qui a pour objet de conserver et de rétablir cette manière d'être qui s'appelle la santé. Les deux parties de l'art qui a pour objet la santé du corps sont la gymnastique et la médecine. Quant à l'art qui a pour objet la santé de l'âme, c'est la politique. Les deux parties de la politique sont la législation et la justice, c'est-à-dire l'art de récompenser et de punir.

%\begin{longtable}[]{@{}lll@{}}
%\toprule()
%Bien & Art de conserver & Art de rétablir \\
%\midrule()
%\endhead Santé du corps & Gymnastique & Médecine \\
%Santé de l'âme & Législation & Justice \\
%\bottomrule()
%\end{longtable}
%
%Pour chacun de ces arts il existe un savoir-faire qui en est la contrefaçon, le simulacre. La contrefaçon de la gymnastique, c'est la cosmétique, qui donne aux corps l'apparence de la santé et de la beauté. La contrefaçon de la médecine, c'est la cuisine. Le cuisinier feint de savoir quels aliments conviennent au corps. Si un tribunal d'enfants devait décider qui du médecin ou du cuisinier s'y connaît le mieux en nourriture, le cuisinier, s'il le voulait, obtiendrait tous les suffrages.
%
%La législation et la justice ont respectivement pour contrefaçon la sophistique et la rhétorique. Le sophiste contrefait le législateur, comme l'orateur contrefait le juge. La distinction entre sophistique et rhétorique est surtout théorique : dans les faits ce sont souvent les mêmes personnes qui pratiquent l'une et l'autre.
%
%\begin{longtable}[]{@{}lll@{}}
%\toprule()
%Bien & Simulacre de l'art de conserver & Simulacre de l'art de rétablir \\
%\midrule()
%\endhead Santé du corps & Cosmétique & Cuisine \\
%Santé de l'âme & Sophistique & Rhétorique \\
%\bottomrule()
%\end{longtable}
